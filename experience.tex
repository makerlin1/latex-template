%----------------------------------------------------------------------------------------
%	WORK EXPERIENCE
%----------------------------------------------------------------------------------------
\lettersection{\faBriefcase \hspace{0.1cm} 竞赛与项目经历}

%----------------------------------------------------------------------------------------
%	IPsoft Technical Executive
%----------------------------------------------------------------------------------------
\begin{tabularx}{1\linewidth}{>{\raggedleft\scshape}p{2.5cm}X}
\gray 竞赛 & \textbf{全国大学生数学建模竞赛} \hfill \textbf{2020年9月}\\
\gray 分工 & \textbf{编程} \hfill \\
\gray 奖项 & \textbf{本科组陕西赛区一等奖} \\
\end{tabularx}

\vspace{2pt}
\textbf{主要负责:}
\begin{itemize}
\item 检索相关外文文献,辅助建模建立数学模型。
\item 使用\textbf{MATLAB}对建立的数学模型进行求解,使模型计算结果能够充分拟合题目所给的数据。
\item 建立对应问题的目标优化模型,使用遗传算法与分尺度逐步搜索法确定相关参数。
\item 使用灵敏度分析研究相关数据不准确或发生变化时所得解的稳定性。
\item 将程序计算出的结果可视化并进行分析。\\
\end{itemize}


%----------------------------------------------------------------------------------------
\begin{tabularx}{1\linewidth}{>{\raggedleft\scshape}p{2.5cm}X}
    \gray 竞赛 & \textbf{华为云无人车大赛} \hfill \textbf{2020年8月}\\
    \gray 分工 & \textbf{目标检测} \hfill \\
    \gray 结果 & \textbf{A榜 第56名 } \\
\end{tabularx}

\vspace{2pt}
\textbf{主要负责:}
\begin{itemize}
\item 调研目标检测方向各类模型与优化技巧。
\item 基于商汤开源的\textbf{MMDetection}工具包,训练目标检测模型\textbf{(Faster-RCNN)}
\item 训练调优与使用相关技巧。\\
\end{itemize}


%	IPsoft Automation Engineer.
%----------------------------------------------------------------------------------------
\begin{tabularx}{1\linewidth}{>{\raggedleft\scshape}p{2.5cm}X}
    \gray 项目 & \textbf{IMDB大规模数据爬取与处理} \hfill \textbf{2020年9月}\\
    \gray 分工 & \textbf{构建\textbf{ip池与多线程爬虫}} \hfill \\
\end{tabularx}

\vspace{2pt}
    \textbf{主要负责:}
    \begin{itemize}
    \item 使用\textbf{bs4}与\textbf{Requests} 设计对应的爬虫代码。
    \item 为解决\textbf{ip}被封的问题,构建代理\textbf{ip}池,实现快速爬取数据。
    \item 利用多台设备开启多线程分布爬取,最终将数据处理储存至\textbf{MySQL}数据库。\\
    \end{itemize}


\begin{tabularx}{1\linewidth}{>{\raggedleft\scshape}p{2.5cm}X}
\gray 竞赛 & \textbf{中国创新挑战赛 智慧教育专题} \hfill \textbf{2020年10月}\\
\gray 赛题 & \textbf{教育手写公式识别} \hfill \\
\gray 结果 & \textbf{ A榜 48名} \\
\end{tabularx}

\vspace{2pt}
\textbf{主要负责:}
\begin{itemize}
\item 查阅相关手写公式识别与自然语言生成等方向的论文资料,
\item 基于深度学习方法,构建图像-字符序列编解码\textbf{(CNN+Transformer)}模型。
\item 在服务器上训练调优模型与记录实验数据。\\
\end{itemize}


\begin{tabularx}{1\linewidth}{>{\raggedleft\scshape}p{2.5cm}X}
    \gray 研究方向 & \textbf{图像内容描述} \hfill \textbf{2020年10月-至今}\\
    \gray 指导老师 & \textbf{冯志玺} \hfill \\
    \end{tabularx}\\
    
    \vspace{2pt}
    \textbf{主要内容:}
    \begin{itemize}
    \item 调研相关\textbf{Image Caption}的文献与最新研究
    \item 精读分析经典论文,复现相关模型。
    \item 在服务器上训练调优模型与记录实验数据。\\
    \end{itemize}
